\subsection{Some terms}


It's difficult to talk about the sounds in language without resorting to various terms that are definitely not in common use. My aim has to give a broad idea of their meanings: accuracy has been sacrificed for simplicity wherever possible. For a much more accurate description, see, e.g., \citet{ladefoged_course_2011}.

\begin{description}[align=left]
\item [obstruent] obstruents \textit{obstruct} the flow of air in some way; they are divided into stops, fricatives, and affricates.
\item [stop] when the flow of air is stopped, then released, there is a small burst of sound; sounds created in this manner are also called \textit{plosives}. For example, the \textit{labial} stops `p' and `b' are created by the closing and opening of the lips.

Not all stops are plosives: aside from nasals (see below), there is also the \textit{glottal stop}, where the air is stopped in the throat (glottis), for example: the pause in ``uh-oh'', or `t' in the middle of a word in a Cockney accent.
\item [fricative] Fricatives are related to stops; instead of completely stopping the airflow, part of it is allowed to release, resulting in a hissing noise.
\item [affricate] Affricates combine a stop with a fricative: the `ch' in ``church'' is a combination of the usual sounds of `t' and `sh'.
\item [nasal] a nasal (or \textit{nasal stop}) involves stopping the flow of air through the mouth, routing it through the nose instead.
\item [voicing] voicing refers to the use of the vocal cords; consonants that use the vocal cords are \textit{voiced}, while those that do not are \textit{voiceless}. Voiceless/voiced pairs in English are p/b, f/v, k/g, t/d, s/z, ch/j, sh/`zh' (the sound of `s' in ``measure''). `th' is complicated, as it represents both voiced and voiceless sounds (`th' in ``thing'' is voiceless, while in ``then'', it's voiced). Some dialects of English also have a voiceless version of `w', as in ``when'' and ``whale''.

Voicing is a somewhat complicated topic in English, as ``voiced'' consonants are only partly voiced, if at all; ``voiced'' consonants in English are shorter, and not aspirated.
\item [aspiration] aspiration is a small burst of air that typically follows voiceless consonants at the start of words, except when following `s'; the `p' (\textipa{[p\super{h}]}) in `pin' is usually aspirated, while the `p' (\textipa{[p]}) in `spin' is not.\footnote{This can be tested by holding a sheet of light paper in front of the mouth.}
\end{description}
