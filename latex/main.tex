\documentclass{article}[11pt]
\usepackage[T1]{fontenc}
\usepackage{tipa}
\usepackage{titling}
\usepackage[utf8]{inputenc}
\usepackage[english]{babel}
\usepackage{hyperref}
\usepackage{csquotes}

%\usepackage{biblatex}
\usepackage{natbib}
\usepackage{enumitem}
\bibliographystyle{unsrtnat}

\setlength{\droptitle}{-6em}

%\addtolength{\oddsidemargin}{-.875in}
%\addtolength{\evensidemargin}{-.875in}
%\addtolength{\textwidth}{1.75in}
%\addtolength{\topmargin}{-.875in}
%\addtolength{\textheight}{1.75in}

\title{Assignment}
\author{Jim O'Regan}
\date{June 2021}

\begin{document}

\maketitle

\section{Introduction}



\section{Background}

%\subsection{Some terms}

%It's difficult to talk about the sounds in language without resorting to various terms that are definitely not in common use. My aim has to give a broad idea of their meanings: accuracy has been sacrificed for simplicity wherever possible. For a much more accurate description, see, e.g., \citet{ladefoged_course_2011}.

%\begin{description}[align=left]
%\item [obstruent] obstruents \textit{obstruct} the flow of air in some way; they are divided into stops, fricatives, and affricates.
%\item [stop] when the flow of air is stopped, then released, there is a small burst of sound; sounds created in this manner are also called \textit{plosives}. For example, the \textit{labial} stops `p' and `b' are created by the closing and opening of the lips.

%Not all stops are plosives: aside from nasals (see below), there is also the \textit{glottal stop}, where the air is stopped in the throat (glottis), for example: the pause in ``uh-oh'', or `t' in the middle of a word in a Cockney accent.
%\item [fricative] Fricatives are related to stops; instead of completely stopping the airflow, part of it is allowed to release, resulting in a hissing noise.
%\item [affricate] Affricates combine a stop with a fricative: the `ch' in ``church'' is a combination of the usual sounds of `t' and `sh'.
%\item [nasal] a nasal (or \textit{nasal stop}) involves stopping the flow of air through the mouth, routing it through the nose instead.
%\item [voicing] voicing refers to the use of the vocal cords; consonants that use the vocal cords are \textit{voiced}, while those that do not are \textit{voiceless}. Voiceless/voiced pairs in English are p/b, f/v, k/g, t/d, s/z, ch/j, sh/`zh' (the sound of `s' in ``measure''). `th' is complicated, as it represents both voiced and voiceless sounds (`th' in ``thing'' is voiceless, while in ``then'', it's voiced). Some dialects of English also have a voiceless version of `w', as in ``when'' and ``whale''.

%Voicing is a somewhat complicated topic in English, as ``voiced'' consonants are only partly voiced, if at all; ``voiced'' consonants in English are shorter, and not aspirated.
%\item [aspiration] aspiration is a small burst of air that typically follows voiceless consonants at the start of words, except when following `s'; the `p' (\textipa{[p\super{h}]}) in `pin' is usually aspirated, while the `p' (\textipa{[p]}) in `spin' is not.\footnote{This can be tested by holding a sheet of light paper in front of the mouth.}
%\end{description}


\subsection{Syllables}

\begin{displayquote}
\textbf{syllable} A unit of speech for which there is
no satisfactory definition. Syllables seem to
be necessary units in the mental organization and production of utterances.~\citep{ladefoged_course_2011}
\end{displayquote}

\citet{goslin_comparing_2007} sort these syllabification methods under two broader categories: legality, and sonority. Methods in the legality category are based on the idea that syllables can only start with consonant sequences that are valid at the starts of words, and end with sequences that are valid ends of words. Sonority methods use a ranking of sonority, or loudness, of phonemes, so that sonority maximally rises towards the vowel, and maximally lowers away from it: in ``dormant'', the liquid `r' is ranked higher in sonority than the nasal `m', so ``dor-mant'' is the best syllabification.

\citet{saussure_course_1959} mentions vocalic fricatives as a criticism of sonoric syllabification; for English, at least, these hold equally for legality-based syllabification: in English, in general, words cannot start with `bz', `ps', or `pf', yet we have interjections `bzz', `psst', `pff'.

``particular'' could be broken into words as ``par tic you Lar'' or ``part ick yule are''.

\subsection{Syllables in ASR}

\citep{fujimura_syllable_1975}

\citet{marchand_automatic_2009} list two major problems to the creation of a gold standard syllabification corpus: disagreement among annotators, and the impracticality of collecting a sufficiently large corpus.

\citet{soltau_neural_2017} trained a speech recognition system with CTC, using whole words as units.


\bibliography{main}
%\bibliographystyle{apa}
%\nocite*{}

\end{document}